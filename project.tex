%% stripped down version of "bare_jrnl.tex" for use in Casey's Circuits I class.
%% Original version has very good comments for use. You should check it out at
%% http://www.ieee.org/publications_standards/publications/authors/authors_journals.html
%% I run GNU/Linux so I downloaded the "Unix LaTeX2e Transactions Style File" package
%% and based my work off of the sample tex file named "bare_jrnl.tex".

% original author info below (this guy's a rockstar for making his comments so easy to use :P)
%% 2007/01/11
%% by Michael Shell
%% see http://www.michaelshell.org/
%% for current contact information.

\documentclass[journal]{IEEEtran}
% make sure "IEEEtran.cls" is in the path of the tex file you are working on

%graphics package for adding images
\ifCLASSINFOpdf
  \usepackage[pdftex]{graphicx}
\else
   \usepackage[dvips]{graphicx}
\fi

%math package for math equations
\usepackage[cmex10]{amsmath}

%float package for putting images where I fucking tell them to go
\usepackage{float}

%for sourcecode
\usepackage{listings}
\lstset{breaklines=true,language=gnuplot,basicstyle=\scriptsize,showspaces=false,showstringspaces=false}

%For hyperlinks
\usepackage{hyperref}

%For colored text
\usepackage{color}

% There are TONS of other packages you can use for various different cases.
% Bibliographies are big in research papers but not important for lab reports
% (I think...?). So I haven't included any bibliography code

\begin{document}

% paper title
% can use linebreaks \\ within to get better formatting as desired
\title{Class-A Power Amplifier Design Proposal\\Team Helios}
\author{Preston~Maness\\Alexander~Kendrick\\Mark~McQuilken\\Jonathan~Wright}

% header
\markboth{Texas State University, Dr. Aslan, EE3350, Final Project}%
{}

% make the title area
\maketitle

% Give the abstract of your lab here
\begin{abstract}
The final project for Dr. Aslan's Fall 2012 Electronics I course was to design a power amplifier for use in audio applications. The goal was to apply knowledge of power supplies (diodes and transformers) and amplifiers (BJT transistors with hybrid pi model) to design a Class-A power amplifier from end to end. Particular emphasis was placed on collaborative design and CAD-assisted verification via Multisim software, as well as having as large a bandwidth as possible. 

Team Helios chose a common emitter to emitter follower design as a base. Choosing to employ a bipolar dual-rail power supply to the system --so as to increase maximum voltage swing and have zero power output in the absence an applied signal (no heat dissipation)-- necessitated the introduction of a third transistor to act as a voltage shift. The result is an amplifier that has an approximate voltage gain of 142 and a power output of approximately five milliwatts across resistive loads varying from 2 to 16 ohms, as per the specification. Efficiency was measured as approximately five percent compared to the DC input. Total cost of materials was determined to be \$17.93 dollars.
\end{abstract}

\tableofcontents
\listoffigures
%hacky. might remove later.
%\newpage

% Split your lab report into sections by calling \section{Section name}
\section{Introduction and Product Specifications}
\IEEEPARstart{T}{he} client, Good Speakers Inc., has commissioned the design of a Class-A Power Amplifier for use in audio applications. The following specifications are provided. 

\begin{itemize}
\item
Input:  $V_{s}=1 mVpk$, $R_{s}=50\Omega$
\item
Output: Speaker that ranges from $2\Omega$ to $16\Omega$.
\item
Onboard 12V DC power supply (bipolar permited)
\item
USB connector for charging purposes (investigate whether this is possible)
\item
High power gain
\item
High efficiency
\item
Minimum cost
\end{itemize}

%\begin{figure}[h!]
%\centering
%\includegraphics[scale=.35]{schematic.png}
%\label{fig_schem}
%\end{figure}

\section{Implications of Specifications}

Given the audio application of the amplifier, one can reasonably assume that the frequency of the input signal will fall between 22 Hz and 22 kHz. While the client desires as wide a bandwidth as possible, the audio bandwidth is the most crucial to meet. The lower bound of 22 Hz would require significantly larger capacitors to place the lower cutoff value at said frequency. Team Helios assumes a lower bound of 100 Hz in calculations, knowing that the effect on these low frequencies past the 100 Hz cutoff is negligible. In addition, the chosen design that uses DC coupling to minimize the need for capacitors has an added effect of reducing attenuation at each cutoff frequency.

The request to investigate possibilities of USB charging was at first interpreted as an indication that the circuit needed to be powered by battery. Further communication with client indicates that this assumption wasn't accurate. The client is interested in the ability to charge the signal source --mp3 player, iPod, etc-- as it is playing. This is possible. However, the charging currents required by USB devices vary considerably, and the ability to charge and operate at the same time is widespread but not universal.

The high power gain and high efficiency requirements are restricted by the request for a Class A amplifier; such amplifiers have a maximum theoretical efficiency of 25\%, with typical designs in the single digits.

\section{Power Supply Design}

The following design is an improvement on the basic models introduced at the start of semester. It takes into account various sources of noise and distortion common to mains input. It also attempts to mitigate common sources of design failure, such as overvoltage. Finally, it further addresses safety concerns with the introduction of arc-resistant fuses and is a step in the right direction for approval by certifiers such as Underwriters Laboratories. 

\begin{figure}[H]
\centering
\includegraphics[scale=.36]{final-psu.png}
\caption{Final PSU Schematic}
\label{fig_psu_schem}
\end{figure}

\subsection{Block Diagram}

From mains input to DC output, there are multiple issues to address. Switching and fusing off mains before hitting the transformer is a given. Along the way, one may address EMI, inrush current, and overvoltage protections for the voltage regulator chips.

\begin{figure}[H]
\centering
\includegraphics[scale=.35]{psu-block.png}
\caption{PSU Block Diagram}
\label{fig_psu_block}
\end{figure}

\section{Rejected Amplifier Designs}

Various approaches and topologies were investigated. From single-transistor to CE-EF to op-amp implementations, each design was examined on a broad level with advantages and drawbacks determined.

\subsection{Transformer Coupled}

This design used a single CE transistor, replacing $R_{C}$ with a transformer to increase efficiency.
The image was located at the second reference listed in the reference section.

\begin{figure}[H]
\centering
\includegraphics[scale=.4]{reject-1.jpg}
\caption{Rejected Transformer Coupling}
\label{fig_reject_1}
\end{figure}


\subsection{CE-EF one rail}

This design used a CE-EF implementation with a single 12 V rail.
\begin{figure}[H]
\centering
\includegraphics[scale=.4]{reject-2.png}
\caption{Rejected CE-EF Single Rail}
\label{fig_reject_2}
\end{figure}

\subsection{Op-Amp Implementation}

This design implemented the specification utilizing Op-Amps.

\begin{figure}[H]
\centering
\includegraphics[scale=.4]{reject-3.png}
\caption{Rejected Op-Amp Implementation}
\label{fig_reject_3}
\end{figure}

\subsection{Comparison Summary}

The following table summarizes the comparisons of these designs. 

\begin{tabular}{|c|p{2.5cm}|c|c|}
\hline
& Advantages & Drawbacks \\
\hline
A & -Single transistor & -Transformers are bulky\\
  & -Peak efficiency of 50\% & -Outdated design\\
\hline
B& -CE-EF gives good impedence matching & -Always drawing power \\
 & -Intuitive & -Hum on bypass cap\\
\hline
C & -Very stable & -Team unfamiliar with Op-Amps\\
\hline
\end{tabular}

The bulkiness of the transformer coupled design removed it as a choice. The hum on the bypass capacitor eliminated the CE-EF one rail design. The Op-Amp implementation, while very stable, was not chosen as team Helios is unfamiliar with Op-Amps.

\section{Selected Amplifer Design}

The following schematic is the chosen design of team Helios. It is comprised of a CE transistor, followed by a complimentary transistor that acts as voltage shift, followed by a Darlington pair to act as the EF to the load.

\begin{figure}[H]
\centering
\includegraphics[scale=.35]{final-schem.png}
\caption{Final Amp Schematic}
\label{fig_amp_schem}
\end{figure}

\subsection{Block Diagram}

\begin{figure}[H]
\centering
\includegraphics[scale=.38]{amp-block.png}
\caption{Amp Block Diagram}
\label{fig_amp_block}
\end{figure}

\subsection{Design Philosophy}

The traditional approach for design, starting with KVL and $I_{C}=\beta I_{B}$, functions well for single-transistor implementations. It is thorough and rigourous and the hybrid pi model is very accurate for the frequency range under consideration. However, this design methodology's heavy dependence on assumptions for system behaviour that can change and are at times interdependent can often lead to line after line of dense calculation that must be revamped continuously for each iteration of the design. Performing such calculations manually is error prone, and writing software to automate the process is outside of the scope of this project. 

Consequently, a work backward approach was taken, starting from arbitrarily selected target output values --for example, require 10 mW output power-- and working backwards to determine requirements at each stage of the amplifier. For example: in the given design, a target output of 5 mW was selected. This implies an RMS voltage on the output. Knowing that this must come from the emitter-follower stage (Darlington pair in the chosen design), one can conclude that the RMS voltage entering the emitter-follower must be greater than $1.4 V$, since $V_{BE}$ is approximated as $0.7 V$. Knowing that the voltage shift gain is approximately one, one can conclude that this voltage is also the input and output of the voltage shift stage and, consequently, that our common emitter (which performs the actual voltage gain) needs to output this. At this point, one may set up the DC conditions on the CE and tweak $R_{E1}$ as necessary to influence the voltage gain.

This fast and loose approach permits a more rapid development process that can always be checked against the more formal mathematics once ballpark values are determined. In our case, design verification was performed using Multisim.

\subsection{Analysis and Calculations}

Given the specification and application, the following are the primary objectives of any design, and each is addressed by the design from Team Helios.

\begin{itemize}
\item
\textbf{Meet Design Requirements --} These must be met first prior to any
improvements on the design.
\item
\textbf{Prevent DC at Load --} Since a speaker is the specified load and speakers
are easily destroyed by DC, keeping DC from reaching the load is a design con
straint that is implied from the customer’s design requirements.
\item
\textbf{Directly Coupled --} There are three choices of coupling our emitter-fol
lower: capacitively coupled, transformer coupling or direct coupling. Capacitive
coupling and transformer-coupling both have serious downsides (bad low-fre
quency response, bulky, expensive, frequency dependence, etc). If it’s possible,
DC-coupling should be used.
\item
\textbf{Non-inverting from input to output --} With an eye toward perfor
mance improvement, better technologies (like opamps) could be used to more
cheaply/compactly improve performance and cost-effectively allow feature
enhancements/additions. Although it’s not impossible to retrofit an inverting stage
with an opamp (there’s always more than one way to achieve desired perfor
mance), it would help ease the design process.
\end{itemize}

\subsubsection{Design Approach}

The following serves as a more detailed explanation of the design philosophy described
above.

\begin{enumerate}
\item
Select a modest goal for the amplifier’s output level: since this was not in the
“specifications”, we chose --through a little trial-and-error-- a 5mW output level
into 2 $\Omega$. Matching the output level with the specified input level yields an AV of
141 (see equation (2), below).
\item
A CE amplifier with bipolar-supply emitter-bias has a shot of the base being close
to ground (ref: pg 241-244 of classroom text). It made sense to start there.
\item
Run a high-enough $I_{E}$ so that a “reasonably”-sized $R_{C}$ will give us the gain we
need.
\item
Assume that $V_{DCLOAD} = 0$.
\item
Find a way to mesh the voltage level at the Darlington pair’s base and the VC both
under quiescent conditions.
\item
If $V_{CEQ}$ is half-way between ground and +12V ($V_{CC}$), the signal can swing $\pm$ 6V.
Given that the signal at the emitter will effectively not vary (we’re supposed to
accommodate an input signal of $1mV_{PK}$), we don’t have to worry about premature
clipping due to $V_{e}$ and $V_{c}$ conflicting. Besides, since we’ll couple the AC voltage
at the collector with the load (through level-shift and the CC Darlington), a
$2.12 V_{RMS}$ signal into 2$\Omega$ is way overkill (the resultant potential power at the load
would be 2W).
\end{enumerate}

\subsubsection{CE Gain Stage}

Picking an $R_{C}$ of 4.99k with a $V_{C}$ of 6V gives an $I_{C}$ of 1.2mA. This yields:

\begin{equation}
r'_{e} = 20.8\Omega
\end{equation}

Ignoring loading on $R_{C}$, this yields a maximum gain of 238. This is almost double the gain that we
need:

\begin{equation}
A_{V} = \frac{V_{OUT}}{V_{IN}} = \frac{\sqrt{P_{OUT}\cdot R_{LOAD}}}{V_{IN}} = \frac{\sqrt{\left(5mW\right)\cdot \left(2\Omega\right)}}{\frac{2}{2\sqrt{2}}mV_{pp}} = 141
\end{equation}

Once this is established, we can select the value for $R_{E}$ to give us “ground” at the base. Working
up from -12V:

\begin{equation}
R_{E} = \frac{12V - 0.7V}{1.2ma} \approx 10k\Omega
\end{equation}

For an $A_{V}$ of 141,

\begin{equation}
R_{E1} = \frac{R_{C}}{A_{V}} - r'_{e} \approx 15\Omega
\end{equation}

This value will be “tweaked” once the whole circuit’s been put together. The final CE circuit along with
$C_{B}$ is given below:

\begin{figure}[H]
\centering
\includegraphics[scale=.6]{CE-stage.png}
\caption{CE Stage}
\label{fig_ce_stage}
\end{figure}

\subsubsection{Output Driver}

The specification is lacking in certain details like maximum input levels. However, given that the input
signal is specified as 0 to 2$mV_{pp}$, we’ll assume that the maximum is $2mV_{pp}$ while the minimum is
0. Adding just a little extra for good measure, we’ll assume the peak current into the load to be
100mA (10mW into 2$\Omega$). In a Class A amplifier, the current either flows into the load or into a
current sink/source. An emitter follower can only source current, not sink it. So for the negative
swing across the load to reach it’s peak, we need a current sink to provide the peak current. A
“pull-down” resistor to the negative rail can do (in a pinch) instead of a transistor-based current
sink:

\begin{equation}
R_{sink} = \frac{12V - 0.2V}{100mA} = 118\Omega
\end{equation}

This will provide our peak negative current to the load, when the emitter follower can’t.

\subsubsection{Bridging the Gap}

We have two remaining issues to meet our stated goals of:

\begin{itemize}
\item
Direct-coupled output, and,
\item
Non-inverting output
\end{itemize}

Assuming that there’s a mechanism that allows us to have the CC emitter at 0V, the base
voltage for the Darlington would be a nominal quiescent voltage of 1.4V. Unfortunately, the col
lector voltage of the CE has a $VC_{EQ}$ of 6V. We need a circuit to shift the voltage from 6 to 1.4 and
to invert the signal phase, like this:

\begin{figure}[H]
\centering
\includegraphics[scale=.6]{shift-stage.png}
\caption{Level-shifter Stage}
\label{fig_ls_stage}
\end{figure}

Although there's a slight amount of gain, you can see that if $R_{2}$ had twice the value as $R_{1}$, the volt
age across $R_{1}$ would be double across R2. This is due to the fact that $V_{IN}$ is constant (ignoring the
AC component) and sets up a fixed current in $R_{1}$ which, in turn, flows through $R_{2}$ causing a volt
age drop equivalent to $V_{R2} = I_{E}\cdot R_{2}$. Stated another way, 6V across 5k$\Omega$ is the same current as
12V across 10k$\Omega$. In this case, we’ll adjust the value of $R_{2}$ so that the q-current set up in $R_{1}$ gives
us a voltage across $R_{2}$ of 1.4V above ground. This is where we expect the base of the Darlington to
roughly be. Thus, the drop across the value for $R_{2}$ shown in Figure 3 will be 13.4V above the neg
ative rail, or 1.4V above ground! Also, this stage is an inverter that takes the inverted signal from
the CE and re-aligns the polarity fed to the non-inverting emitter followers to match the input sig
nal.

This resistor value can be “tweaked” to give us exactly i$0 V_{DC}$ across the load by using a multi-turn
potentiometer in series with $R_{2}$. This solution may be good for a breadboard/prototype, but would
not be good solution for a real product because:

\begin{itemize}
\item
The pot’s value will drift with time/temperature/mechanical vibration, and,
\item
Each amplifier would have to be individually trimmed
\end{itemize}

To avoid this scenario, we can use an ingenious opamp stage called a ``servo'' amplifier.

\subsubsection{Final Touch}

The operation of a servo amp is simple: configured as an inverting integrator, an opamp is con
nected to a node of interest to adjust and the output (-180$^{\circ}$ out of phase) is injected into another
node that can control the node that its monitoring. Figure 4 shows a typical servo application:

\begin{figure}[H]
\centering
\includegraphics[scale=.6]{servo-app.png}
\caption{Use of Op-Amps for Feedback Control}
\label{fig_op_amp_demo}
\end{figure}

In the above figure, U1A is an inverting amplifier of gain $-\frac{R_{5}}{R_{2}}$. Since we’re feeding 15V into R2, 
the output of U1A (when S1 is open) is -1.5V. Closing S1 causes the servo to sense the -1.5V and apply
slowly increasing opposite voltage. The voltage climbs to the point that the output of U1A become
0V. Any signal with a frequency lower than the servo’s cutoff frequency will be corrected...as long
as U1B’s output can swing sufficient large to adjust U1A’s offsets via the voltage divider consist
ing of $R_{1}$ \& $R_{4}$.

All of these stages, along with the Darlington EF, form Team Helios' final schematic given at the
start of this section.

Observe how $R_{E1}$ has been tweaked as a result of design verification using various simulation tools of
multisim. As well, observe how $R_{E}$ on the level shifter is split when a servo is employed.

\section{Cost Analysis of Selected Design}

A bill of materials is below:

\begin{enumerate}
\item
3 BJT\_NPN, 2N2222A, \$1.02
\item
1 BJT\_PNP, 2N3906, \$0.42
\item
1 OPAMP, TL072ACD, \$0.65
\item
1 FWB, 1B4B42, \$0.55
\item
1 SPST, switch, \$1.15
\item
1 VOLT REG, LM317LM, \$0.78
\item
1 VOLT REG, LM337LM, \$0.86
\item
2 DIODE, 1N4004, \$0.86
\item
1 FUSE, 0.5 AMP, \$0.54
\item
1 CT TRANSFORMER, 6:1, \$11.1
\end{enumerate}

\textbf{TOTAL: \$17.93}

Resistor and capacitor costs are relatively negligible compared to the various transistors, op-amp, and transformer. 
Pricing for the above components was obtained from digikey, and is based on a minimum order of one part.
Additional savings are possible depending on the quantity of amplifiers the client wishes to produce.
As well, the final cost may increase or decrease depending on the choice of manufacturer and any need to
ship the amplifiers to multiple locations.

\section{Conclusion}

The design provided by Team Helios meets and exceeds customer expectations. It accounts for various 
sources of circuit failure and builds in additional stability with its DC coupled CE-EF amplifier and
Op-Amp servo feedback.

\section{References}

\begin{enumerate}
\item
T. L. Floyd, Electronic Devices, PrenticeHall, 9 th. Ed., 2012.
\item
http://www.circuitstoday.com/transformer-coupled-class-a-power-amplifier
\end{enumerate}

\end{document}
