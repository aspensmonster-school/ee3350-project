%% stripped down version of "bare_jrnl.tex" for use in Casey's Circuits I class.
%% Original version has very good comments for use. You should check it out at
%% http://www.ieee.org/publications_standards/publications/authors/authors_journals.html
%% I run GNU/Linux so I downloaded the "Unix LaTeX2e Transactions Style File" package
%% and based my work off of the sample tex file named "bare_jrnl.tex".

% original author info below (this guy's a rockstar for making his comments so easy to use :P)
%% 2007/01/11
%% by Michael Shell
%% see http://www.michaelshell.org/
%% for current contact information.

\documentclass[journal]{IEEEtran}
% make sure "IEEEtran.cls" is in the path of the tex file you are working on

%graphics package for adding images
\ifCLASSINFOpdf
  \usepackage[pdftex]{graphicx}
\else
   \usepackage[dvips]{graphicx}
\fi

%math package for math equations
\usepackage[cmex10]{amsmath}

%float package for putting images where I fucking tell them to go
\usepackage{float}

%for sourcecode
\usepackage{listings}
\lstset{breaklines=true,language=gnuplot,basicstyle=\scriptsize,showspaces=false,showstringspaces=false}

%For hyperlinks
\usepackage{hyperref}

% There are TONS of other packages you can use for various different cases.
% Bibliographies are big in research papers but not important for lab reports
% (I think...?). So I haven't included any bibliography code

\begin{document}

% paper title
% can use linebreaks \\ within to get better formatting as desired
\title{Class-A Power Amplifier Design Proposal\\Team Helios}
\author{Preston~Maness\\Alexander~Kendrick\\Mark~McQuilken\\Jonathan~Wright}

% header
\markboth{Texas State University, Dr. Aslan, EE3350, Final Project}%
{}

% make the title area
\maketitle

% Give the abstract of your lab here
\begin{abstract}
The final project for Dr. Aslan's Fall 2012 Electronics I course was to design a power amplifier for use in audio applications. The goal was to apply knowledge of power supplies (diodes and transformers) and amplifiers (BJT transistors with hybrid pi model) to design a Class-A power amplifier from end to end. Particular emphasis was placed on collaborative design and CAD-assisted verification via Multisim software. 

Team Helios chose a common emitter to emitter follower design as a base. Choosing to employ a bipolar dual-rail power supply to the system --so as to increase maximum voltage swing and have zero power output in the absence an applied signal (no heat dissipation)-- necessitated the introduction of a third transistor to act as a voltage shift. The result is an amplifier that has an approximate voltage gain of 142 and a power output of approximately five milliwatts across resistive loads varying from 2 to 16 ohms, as per the specification. Efficiency was measured as approximately five percent compared to the DC input. Total cost was determined to be X dollars. Lead times from manufacture to customer delivery are conservatively given at two months.
\end{abstract}

\tableofcontents

% Split your lab report into sections by calling \section{Section name}
\section{Introduction and Product Specifications}
\IEEEPARstart{T}{he} client, Good Speakers Inc., has commissioned the design of a Class-A Power Amplifier for use in audio applications. The following specifications are provided. 

\begin{itemize}
\item
blah
\item
blah
\item
blah
\item
blah
\end{itemize}

%\begin{figure}[h!]
%\centering
%\includegraphics[scale=.35]{schematic.png}
%\label{fig_schem}
%\end{figure}

\section{Implications of Specifications}

Given the audio application of the amplifier, one can reasonably assume that the frequency of the input signal will fall between 22 Hz and 22 kHz. While the client desires as wide a bandwidth as possible, the audio bandwidth is the most crucial to meet. The lower bound of 22 Hz would require significantly larger capacitors to place the lower cutoff value at said frequency. Team Helios assumes a lower bound of 100 Hz in calculations, knowing that the effect on these low frequencies past the 100 Hz cutoff is negligible. In addition, the chosen design that uses DC coupling to minimize the need for capacitors has an added effect of reducing attenuation at each cutoff frequency.

asdf

\section{Power Supply Design}

The following design is an improvement on the basic models introduced at the start of semester. It takes into account various sources of noise and distortion common to mains input. It also attempts to mitigate common sources of design failure, such as inrush current and overvoltage. Finally, it further addresses safety concerns with the introduction of arc-resistant fuses and is a step in the right direction for approval by certifiers such as Underwriters Laboratories. 

\begin{figure}[H]
\centering
\includegraphics[scale=.4]{final-psu.png}
\label{fig_psu_schem}
\end{figure}

\subsection{Block Diagram}

asdf

\subsection{Considerations}

asdf

\section{Rejected Amplifier Designs}

Various approaches and topologies were investigated. From single-transistor to CE-EF to op-amp implementations, each design was examined on a broad level with advantages and drawbacks determined.

\subsection{First}

my transformer design

\subsection{Second}

similar design to chosen one that doesn't use bipolar supply (increases hum). 

\subsection{Third}

one of the designs that used lots of op-amps (usable, but lack of familiarity with op-amps resulted in using the selected design)

\subsection{Comparison Summary}

Advantages Drawbacks 

\section{Selected Amplifer Design}

The full schematic is below

\begin{figure}[H]
\centering
\includegraphics[scale=.4]{final-schem.png}
\label{fig_amp_schem}
\end{figure}

\subsection{Block Diagram}

adsf

\subsection{Design Approach}

The traditional approach for design, starting with KVL and $I_{C}=\beta I_{B}$, functions well for single-transistor implementations. It is thorough and rigourous and the hybrid pi model is very accurate for the frequency range under consideration. However, this design methodology's heavy dependence on assumptions for system behaviour that can change and are at times interdependent can often lead to line after line of dense calculation that must be revamped continuously for each iteration of the design. Performing such calculations manually is error prone, and writing software to automate the process is outside of the scope of this project. 

Consequently, a work backward approach was taken, starting from arbitrarily selected target output values --for example, require 10 mW output power-- and working backwards to determine requirements at each stage of the amplifier. For example: in the given design, a target output of 5 mW was selected. This implies an RMS voltage on the output. Knowing that this must come from the emitter-follower stage (Darlington pair in the chosen design), one can conclude that the RMS voltage entering the emitter-follower must be greater than $1.4 V$, since $V_{BE}$ is approximated as $0.7 V$. Knowing that the voltage shift gain is approximately one, one can conclude that this voltage is also the input and output of the voltage shift stage and, consequently, that our common emitter (which performs the actual voltage gain) needs to output this. At this point, one may set up the DC conditions on the CE and tweak $R_{E1}$ as necessary to influence the voltage gain.

This fast and loose approach permits a more rapid development process that can always be checked against the more formal mathematics once ballpark values are determined. In our case, design verification was performed using Multisim.

\subsection{Calculations}

Still haven't gotten any calculations. Granted, the above does a good job on the philosophizing.

\subsection{Considerations}

adsf

\section{Cost Analysis of Selected Design}

asdf

\section{Conclusion}

asdf

\section{Derived Component Values}

This section documents the general process for finding ideal component values to meet the specifications. In each of these cases it is assumed $\omega_{o}$ is in terms of rad/s. In each case, one started by selecting an available value for one of the components, and then finding the available value for the subsequent components that would bring the resonant frequency closest to that requested.

\subsection{RC high-pass}

For an RC high-pass circuit, the following relationship may be exploited to determine proper R and C values:

\begin{align*}
\omega_{o} = \frac{1}{RC}
\end{align*}

A choice of C = 1$\mu$F led to a value of R = 27$\Omega$. This selection provides a resonant frequency of 37,037 rad/s (5894 Hz).

\subsection{RL low-pass}

For an RL low-pass circuit, the following relationship may be exploited to determine proper R and L values:

\begin{align*}
\omega_{o} = \frac{R}{L}
\end{align*}

A choice of L = 4.7 mH led to a value of R = $37\Omega$. This selection provides a resonant frequency of 2765.95 rad/s (440.2 Hz).

\subsection{RLC band-pass}

For an RLC band-pass circuit, the following relationships may be exploited to determine proper R, C, and L values:

\begin{align*}
\omega_{o} &= \sqrt{\frac{1}{LC}}\\
\omega_{2} &= \frac{\frac{R}{L} + \sqrt{\left(\frac{R}{L}\right)^{2} + 4\omega^{2}}}{2}\\
\beta &= \left(\omega_{2}-\omega_{o}\right)\cdot 2\\
Q &= \frac{\omega_{o}}{\beta}
\end{align*}

Start by selecting an available inductor. Then find an appropriate available capacitor so as to approach the desired resonant frequency. Once one has L and C values, utilize the second equation above, in conjunction with the third and fourth, to find a resistor that will narrow the bandwidth to a range  sufficiently small to meet the minimum Q-factor. One is essentially using brute force over the available resistor components to find $\omega_{2}$, and then checking whether the resulting bandwidth (and thus resulting Q-factor) is appropriate. The following values exceed the specification and approach a Q-factor of 15:

\begin{enumerate}
\item
L = 4.7mH
\item
C = 10nF
\item
R = 47$\Omega$
\item
$\omega_{o}$ = 145,864 rad/s (23,215 Hz)
\item
Q = 14.34
\end{enumerate}

\subsection{RLC band-stop}

A band-stop filter may be designed as though one were creating a band-pass filter. One must still wire the band-stop filter components correctly though. The same relationships and process as above may be utilized to arrive at component values.

\begin{enumerate}
\item
L = 2.2mH
\item
C = 10nF
\item
R = 27$\Omega$
\item
$\omega_{o}$ = 213,200 rad/s (33,931 Hz)
\item
Q = 17.37
\end{enumerate}

\section{Experimental Procedure}

% Use enumerate to give numbered lists
\begin{enumerate}
\item
Construct each filter above on the NI ELVIS board.
\item
Apply the proper wiring from pin outs for the function generator (the source for the circuit) and the probes utilized by the bode analyzer. It is recommended that one uses A1 for the source and A0 for the response.
\item
Engage the bode analyzer. Adjust the step, minimum frequency, and maximum frequency options as necessary in order to generate a usable plot for analysis.
\item
Should the bode analyzer throw a warning regarding "OVERFLOW," restart the bode analyzer.
\end{enumerate}

\section{Collected Data and Analysis}

\subsection{RC High-pass}


\begin{tabular}{|c|c|c|c|}
\hline
Value & Expected & Measured & \% error \\
\hline
$\omega_{o}$ (Hz) & 5894 & 2747 & -53.4\\
\hline
phase (deg) & 45 & 41.096 & -8.67\\
\hline
\end{tabular}

The measured resonant frequency is much lower than anticipated. Further inspection of the components revealed that a 62$\Omega$ resistor had actually been used. Knowing this, the expected value would have been 2567 Hz, much closer to the measured value and with a much smaller percent error (-6.55\%).

\subsection{RL Low-pass}


\begin{tabular}{|c|c|c|c|}
\hline
Value & Expected & Measured & \% error \\
\hline
$\omega_{o}$ (Hz) & 440 & NA & NA\\
\hline
phase (deg) & 45 & NA & NA\\
\hline
\end{tabular}

As one can see from the above graph, the filter never exceeded a -6 dB gain, regardless of settings given to the bode analyzer. It is as though the gain has been shifted downward by 6 dB across all frequencies. As well, the line is significantly less concave than the high-pass filter. A source for this behaviour was unable to be determined.

\subsection{RLC Band-Pass}


\begin{tabular}{|c|c|c|c|}
\hline
Value & Expected & Measured & \% error \\
\hline
$\omega_{o}$ (Hz) & 23,215 & 22,288 & -3.99\\
\hline
phase (deg) & 0 & -2.07 & NA\\
\hline
\end{tabular}

The bode plots behaved as expected both in general shape and values returned. Given that the expected phase degree is zero, one cannot use the typical formula for percent error. However, the small difference indicates that the phase plot was behaving as expected. 

\subsection{RLC Band-Stop}


\begin{tabular}{|c|c|c|c|}
\hline
Value & Expected & Measured & \% error \\
\hline
$\omega_{o}$ (Hz) & 33,931 & 32,804 & -3.32\\
\hline
phase (deg) & 0 & -4.53 & NA\\
\hline
\end{tabular}

Once again the general shape is as expected. As well, the values are well within the 5\% margins requested.

\section{Conclusion}

The general behaviour was confirmed for three of the four basic filters. Several groups reported trouble with the low-pass filter, regardless of its RL or RC nature. Errors exceeding 5\% are expected, given that the tolerance on the resistors alone is 10\% on either side of the listed value. However, the band-pass and band-stop were both within tolerance even with such components.

\end{document}
